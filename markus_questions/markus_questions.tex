% Created 2021-08-26 Thu 16:02
% Intended LaTeX compiler: pdflatex
\documentclass[a4paper]{article}
\usepackage[utf8]{inputenc}
\usepackage[T1]{fontenc}
\usepackage{graphicx}
\usepackage{grffile}
\usepackage{longtable}
\usepackage{wrapfig}
\usepackage{rotating}
\usepackage[normalem]{ulem}
\usepackage{amsmath}
\usepackage{textcomp}
\usepackage{amssymb}
\usepackage{capt-of}
\usepackage{hyperref}
\usepackage{listingsutf8}
\usepackage{minted}
\usepackage[margin=0.5in]{geometry}
\usepackage{parskip}
\author{Marco Hassan}
\date{\today}
\title{Answer to Markus Kalisch Questions on Bayesian Networks Software Packages}
\hypersetup{
 pdfauthor={Marco Hassan},
 pdftitle={Answer to Markus Kalisch Questions on Bayesian Networks Software Packages},
 pdfkeywords={},
 pdfsubject={},
 pdfcreator={Emacs 27.2.50 (Org mode 9.4.4)}, 
 pdflang={English}}
\begin{document}

\maketitle
\setlength\parindent{0pt}

\section{Question 1: On Hugin}
\label{sec:org80d48e4}

\begin{quote}
I understand that hugin is the top choice for real world
applications in idustry, is that correct ?
\end{quote}

It is for sure one of the most known software packages. I
encountered it many times in multiple papers I came across when
doing research - so even in academic research it seems to be the
default go-to software package.

However, I think that it is not the most established due to its
completeness or strength but rather due to historical reasons as
being one of the first developed solid piece of software dealing
with Bayesian Networks - see in this sense the \href{https://d1wqtxts1xzle7.cloudfront.net/49642660/HUGIN\_-\_A\_Shell\_for\_Building\_Bayesian\_Be20161016-6003-9fm92w.pdf?1476636550=\&response-content-disposition=inline\%3B+filename\%3DHUGIN\_A\_Shell\_for\_Building\_Bayesian\_Beli.pdf\&Expires=1629908608\&Signature=bhntyNgHmv2Mq-QgPlJOeOObSPuVYVft5MJ0UckslCY8wU80gRSEgFJ63-TNFAj4Wz3HdaqYSv4AkrGKOvH0obgO-FdGYXV8vp\~ve6thEpUAK2M7bD\~F-JNV82h33wd1exhZS03a9JYjt7nRI52q7VokU8h9XOqBKuWZcBT80zOWPWrzp4Y5pTK2-2MFeWfua8iuEDH97ljdHhFe3z\~dOn4SvJ-Afnl\~N7v-jjtevpgME3HLjcTZ23ITUHa8LTP85CatjPT\~i5cFkvA8kxStu-7LkgvqFydLGN-UEdlnkCwGRRrhMuT0WRzLCw\~wjJnukGYQS4Gic\~T4uhISqSUiGQ\_\_\&Key-Pair-Id=APKAJLOHF5GGSLRBV4ZA}{1989 paper} presenting
it. On the top of it, it is very well documented with tons of
papers/\href{https://link.springer.com/chapter/10.1007/978-3-540-45062-7\_49}{books} written on it by now.

I think that, in this sense, it got a lot of traction as being that
present in the literature. There are a couple of other alternatives
that, to my mind, are as good. Moreover, once you move away from the
typical tasks of Bayesian Networks the software falls short and it
is necessary to rely on more niche software. See for instance our
application of parameter learning in the case of uncertain evidence.


\section{Question 2: On GeNIe}
\label{sec:org90c161e}

\begin{quote}
I heard that GeNIe (\url{https://www.bayesfusion.com/}) might be a good
choice in the academic setting. Can you or your supervisor comment
on this ?
\end{quote}

Unfortunately, my knowledge of it is too little. It looks as another
viable option but it would take me more time to double check.

From the documentation it looks as well established and has a
similar UI to Hugin. Furthermore, it looks as if there is a lot of
abstraction such that the user focuses in the nitty gritty
implementation rather than on the algorithm behind. Also the SMILE
API seems to be quite high-level without giving too much the
possibility to tackle advanced cases and operate on the algorithms
working behind it. It makes sense given its license based nature.

Finally, as far as my understanding goes, both Hugin and GeNIe cover
the main classical issues when dealing with Bayesian Networks but do
not cover extensively edge cases where the mathematical theory for
it has already been outlined. In this sense, for instance in the
case of uncertain evidence, GeNIe provides similar functions to
Hugin. Both perform similar operations implementing inference in the
case of likelihood evidence via Pearl's virtual method but never
go as far as dealing with inference in the case of probabilistic
evidence.

Overall, yes. I think that GeNIe and its academic version could have
been/can be used as a valuable substitute to Hugin for performing
the simulation exercise in order to test the implemented algorithm
in Merlin. From the documentation of it and checking at the SMILE
programming API, I think that it might even be more exhaustive than
Hugin.

Nonetheless, I guess that the most suitable option for my
application would be still to obtain a Hugin license as there is
already the code base to run the necessary simulations in Hugin via
its API and a pipeline to pass the result and perform the parameter
learning exercise via the Merlin Engine. That would save a couple
of weeks of work for testing the algorithms.

\section{Question 3: On the Merlin Engine}
\label{sec:org5796e4d}

\begin{quote}
And how does the "merlin engine" fit in (I just know this from
rockets until now ;): From the github description I understand that
this is a stand-alone program for all sorts of BayesNet applications
in C++, right ?
\end{quote}

Correct. Merlin is a stand-alone program through which it is
possible to solve a wide range of Bayesian Networks tasks - mainly
in the domain of parameter learning and network inference. Structure
learning is completely missing in the engine and is not covered as a
Bayesian Network challenge in the source code.

From my personal experience the strength of the engine is rather in
the inference task than in the parameter learning task. In this
sense multiple inference techniques were implemented in the engine,
raging from the most classic message passing algorithms to
approximate inference algorithms. A shortcoming of the engine is, in
this sense, the fact that in the EM-algorithm, and adaptations of
it, it is possible to perform inference just using junction tree
inference algorithms. This limits the extent to which parameter
learning can be applied to some broader class of Bayesian Networks -
say for instance linear Gaussian Networks etc.

On the question on where the algorithm fits in the general picture I
would therefore say the following: while a dozen of engines were
developed to deal with Bayesian Networks there are still quite some
differences among them and it seems that there was still no
convergence to a predominant software platform. I think that this
mostly due to the difficulty of applying such methods and the fact
that the most mature software packages are not-open source and
require in fact some licensing. A good paper that gives the overview
over such fragmented landscape is \href{https://link.springer.com/article/10.1007/s10489-015-0678-6}{Mrad et all (2015)}. There you can
find what major software packages cover in the case of uncertain
evidence. You can see that, for instance, - as said - both Hugin
and GeNIe do not cover inference in the case of probabilistic
evidence. The only established software package that seems to do
that is \url{https://www.bayesia.com/}. In any case none of the
established software embeds the inference step in the case of
uncertain evidence into the EM-algorithm for performing parameter
learning. This was in fact the major contribution there at Research
and the reason to extend the merlin engine. I understood that they
had a client in the manufacturing space where such a solution was
needed and therefore they started implementing such a solution and
subsequently \href{https://ojs.aaai.org/index.php/AAAI/article/view/17447}{published their paper on it}. The paper I am expanding
on so to say.

I guess that a similar picture would result when looking at similar
specialized subfields of different software bundles. I.e. I think
that there are a couple of niche software bundles such as merlin
that grow out of some practical need that was not covered by
classical software bundles such as Hugin and GeNIe and due to the
non-open source nature of them they were not built on the top of
such more mature software bundles yet. However, note that such niche
software bundles do not aim to be as user-friendly and do not cover
every case of treating Bayesian Networks as the more widespread
ones. Think for instance at merlin where there is no nice UI to work
with Bayesian Networks as well as no simulation APIs, no APIs for structural learning, no
game-theory utility based APIs, no opportunity to implement
dynamical Bayesian Networks etc.

I conclude by noting that the integration of Hugin and Merlin I
mentioned in my previous mail was for the sake of testing the
implemented algorithms in the Merlin engine. Nonetheless, Merlin is,
as said, a stand-alone software.

In this sense, while it is theoretically possible to extend the
Merlin engine to perform simulations out of it, there is no API to
this stage to draw samples given a network parameterization. Such a
task was performed by Research by leveraging the Hugin software and
then running the Merlin Algorithms on such simulated data to get a
feeling on the difference in performance when using the proposed
algorithms for parameter estimation in comparison to the classical ones.


\section{Question 4: On R Packages}
\label{sec:org6d87385}

\begin{quote}
Is there any reasonable R package to do at least small scale
applications (what would "small" and "large" mean in this context?)
\end{quote}

I did not look into it to be honest as it was clear from the 
onset that we were developing on merlin and the work of the research
published in \href{https://ojs.aaai.org/index.php/AAAI/article/view/17447}{their paper}.

Nonetheless, with all of that open source C++ code dealing with it,
I guess that there are a tons of wrappers that have been implemented
in R. I guess that the performance of them would be similar to the
C++ counterparts and that "small" and "large" would depend on the
exact task that one aims to perform and its computational
burden. In any case, given that the burden of the tasks generally
increase in an exponential way in the number of variables in the
network - I guess that the smaller the network - the better it
is. From my personal experience in the case of small networks of
around 6-10 variables all of the algorithms works smoothly. I guess
that sound studies have been published extensively covering the
topic. I did not cover extensively such a topic as mentioned in the
conclusion of my thesis as this was not the focus of it.

A final solution when dealing with too large networks is the one of
trying to splitting up the problem by leveraging the concept of
conditional Bayesian Networks. 
\end{document}
