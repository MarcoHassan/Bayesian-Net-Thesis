% Created 2021-04-11 Sun 17:57
% Intended LaTeX compiler: pdflatex
\documentclass[11pt]{article}
\usepackage[utf8]{inputenc}
\usepackage[T1]{fontenc}
\usepackage{graphicx}
\usepackage{grffile}
\usepackage{longtable}
\usepackage{wrapfig}
\usepackage{rotating}
\usepackage[normalem]{ulem}
\usepackage{amsmath}
\usepackage{textcomp}
\usepackage{amssymb}
\usepackage{capt-of}
\usepackage{hyperref}
\usepackage{listingsutf8}
\usepackage{minted}
\usepackage{arxiv}
\author{Marco Hassan}
\date{\today}
\title{}
\hypersetup{
 pdfauthor={Marco Hassan},
 pdftitle={},
 pdfkeywords={},
 pdfsubject={},
 pdfcreator={Emacs 27.1.91 (Org mode 9.4.4)}, 
 pdflang={English}}
\begin{document}

\newtheorem{theorem}{Theorem}

\title{Parameter Learning in Bayesian Networks under Uncertain Evidence  \textendash  \ An Exploratory Research.}
\author{
  Marco Hassan 	           	\\
  Zurich, CH		\\
  \\
  \\
  Master Thesis \\
  Presented to the Eidgenossische Teschnische Hochschule Zurich \\
  In Fulfillment Of the Requirements for \\ 
  the Master of Science in Statistics \\
  \\
  Supervisor: PhD. Radu Marinescu \\
  Co-Supervisor: Dr. Markus Kalisch \\
  %% examples of more authors
  %% \AND
  %% Coauthor \\
  %% Affiliation \\
  %% Address \\
  %% \texttt{email} \\   
  %% \And
  %% Coauthor \\
  %% Affiliation \\
  %% Address \\
  %% \texttt{email} \\
  %% \And
  %% Coauthor \\
  %% Affiliation \\
  %% Address \\
  %% \texttt{email} \\
}

\begin{article}

\maketitle

\newpage

\tableofcontents

\newpage

\section{{\bfseries\sffamily TODO} Bayesian Networks Overview and Definition \& Literature Review}
\label{sec:org890149a}

\section{{\bfseries\sffamily TODO} Learning under Complete Evidence}
\label{sec:org559df44}

\section{{\bfseries\sffamily TODO} Types of Uncertain Evidence}
\label{sec:org6f0678e}

\section{{\bfseries\sffamily TODO} on the nastiness of the Likelihood function}
\label{sec:org0d7b5a5}

all good properties are gone when working with partially observed
data. write down why here. 


\section{The Mathematics of the EM}
\label{sec:orgb0135ab}
As discussed by \cite{koller2009probabilistic} it is possible to frame
the EM as a coordinate ascent optimization of an energy function we
will define next. Given such perspective we will be able to prove the
following theorem

\begin{theorem}\label{thm:one}
Write here formally that the likelihood improves at each iteration step
\end{theorem}

Consider the following energy function:

\begin{equation} \label{eq:energy_functional}
F[P(X), Q] = E_Q[log (\tilde{P}(X))] + H_Q (X)
\end{equation}

Where \(\tilde{P}\) is an unnormalized state probability \(P =
  \frac{\tilde{P}}{Z}\) and \(H_Q\) is the entropy of the observed
particles. 

Using such energy functional \ref{eq:energy_functional} it is possible
to re-express the logarithm of the normalizing constant \(Z\) as
follows:

\begin{equation} \label{eq:energy_refurmolation}
log (Z) = F[P, Q] + D (Q||P)
\end{equation}  

where \(D(Q||P)\) is the Kullback–Leibler divergence, or relative
entropy.

We will choose next the following distribution for the particle
distribution:

\begin{equation} \label{eq:particle_distribution}
P (H | D, \theta) =   \frac{P (H, D| \theta)}{P (D| \theta)}
\end{equation}

With this choice it becomes clear that \(Z (\theta) = P (D|
  \theta)\) and \(\tilde{P} = P (H, Do| \theta)\). It
follows then immediately that given such probability function we
can compute the likelihood of realizations \(\mathscr{D}, \mathscr{H}\):

\begin{align} \label{eq:likelihood_particle}
\mathscr{L} (\theta: \mathscr{D}, \mathscr{H}) =& \  P (\mathscr{H}, \mathscr{D}| \theta)\\
\mathscr{L} (\theta: \mathscr{D}) =& \ P (\mathscr{D}| \theta)
\end{align}

where \(\mathscr{D}\) represents the observed evidence and
\(\mathscr{H}\) the missing evidence.

Such that using \ref{eq:energy_refurmolation} we can get to the
log-likelihood of the observed data in the following way:

\begin{align} \label{eq:likelihood_energy_functional_relation}
l (\theta: \mathscr{D}) =& \  F_D[\theta, Q] + D (Q (\mathscr{H}) || P (\mathscr{H}| \theta, \mathscr{D})) \\
l (\theta: \mathscr{D}) =& \  E_Q[l (\theta: \mathscr{D}, \mathscr{H})]+ H_Q (\mathscr {H}) + D (Q (\mathscr{H}) || P (\mathscr{H}| \theta, \mathscr{D}))
\end{align}

The above are two fundamental equations. It is in fact
straightforward to see that as both the relative entropy as well as
the entropy are non-negative the log-likelihood on the left hand
side above is an upper bound for the energy functional and the expected
log-likelihood relative to Q, for any choice of Q.

Moreover it is straightforward to see in the above that choosing the
Q-measure as \(P (H| D, \theta)\) the relative term
fades away such that the entropy term is the overall measure on the
difference between the expected log-likelihood and the real
log-likelihood. It is in fact clear that in such a case the
log-likelihood and the energy functional are the one and the same
thing.

In this sense the relation between the energy functional and the
log-likelihood is clear and we can think of the EM-algorithm as a
coordinate ascent optimization of the energy functional. To see this
consider the E-step and M-step as follows.

\subsection{The Expectation Step}
\label{sec:org7e14f58}

Consider the first coordinate ascent - Q, keeping \(\theta\)
fixed. We look for \(\operatorname*{argmax}_{Q} F_D[\theta, Q]\). It
is then immediate that:

\begin{align} \label{eq:q_optimum}
Q^* =& \ P (\mathscr{H}|\mathscr{D}, \theta) \\
F_D[\theta, Q^*] =& \ l (\theta: \mathscr{D}) \\
F_D[\theta, Q^*] \geq& \ F_D[\theta, Q]
\end{align}

The reasoning on why the above is the actual searched maximum
argument is the following: You have in general an upper bound on the
energy functional given by log-likelihood. If you now choose the
distribution Q in the way described above you know that you have
reached the upper bound and that such upper bound is tight. I.e. it
is straightforward to see that your are at the maximum for a given
\(\theta\).

Note that choosing \(Q^*\) you are in fact choosing the probability
density by which you are going to weight the synthetically created
complete data sets in your E-step, so that you can in fact
interpret the E-step as the step involving the maximization of the
energy functional along the Q coordinate.

\subsection{The Maximization Step}
\label{sec:org1af855b}

This is the second coordinate ascent - \(\theta\). Here we look
towards \(\operatorname*{argmax}_{\theta} F_D[\theta, Q]\).

It follows then the following quoting from
\cite{koller2009probabilistic}:

"Suppose Q is fixed, because the only term in F that involves \(\theta\) is
\(E_Q[l (\theta: \mathscr{D}, \mathscr{H})]\), the maximization is
equivalent to maximizing the expected log-likelihood."

It follows now that given the linearity of expectation it is
possible to take the expectation of the sufficient statistics and
maximizing for \(\theta\).

This is in fact exactly the standard M-step of the EM algorithm so
that we can interpret the M-step as the coordinate ascent along
the second axis. 

Summarizing, by the fact that at each step the energy functional is
optimized such that it increases it follows from proposition
\ref{eq:likelihood_energy_functional_relation} that the
log-likelihood increases such that theorem \ref{thm:one} is proved.


\section{Bayesian Parameter Learning}
\label{sec:orgcc3b816}

A natural question that arises is whether it is possible to
generalize the extended algorithm proposed by \cite{Mrad_2015} to the
case of Bayesian Parameter Learning.

Recall that in Bayesian statistics rather than treating the
parameters of interest as fixed but unknown you treat them as random
variables themselves.

You would then specify a prior, i.e. a probability distribution, for
the data governing process of the parameters. This can be either a
non-informative prior or a prior based on your domain knowledge
expertise.

Such prior distribution would then be updated upon the arrival of
new observations according to the well known Bayes Rule. The result
is an updated posterior distribution from which you can compute your
statistics of interest.


\begin{equation} \label{eq:bayes_formula}
P (\theta | \mathscr{D}) = \frac{P (\mathscr{D} | \theta) * P(\theta)}{P (\mathscr{D})} 
\end{equation}

It is straightforward to see that that the posterior is proportional
to a likelihood term \(P (\mathscr{D} | \theta)\) multiplied by the
prior distribution.

It is clear then, that depending on how you want to leverage the
information of your posterior you would require a different
mathematical exercise. I.e. in case you want to use as your
point estimate of choice the expected value you would need an
integration exercise and similar reasonings can be done for the
other metrics.

Another way you can set your parameters is by choosing the most
likely point estimate. This is the maximum a posteriori point
estimate and is defined in mathematical terms as follows:

\begin{align} 
\tilde{\theta} =& \operatorname*{argmax}_{\theta} \frac{P (\mathscr{D} | \theta) * P(\theta)}{P (\mathscr{D})} \nonumber\\
\tilde{\theta} =& \operatorname*{argmax}_{\theta} P (\mathscr{D} | \theta) * P(\theta)\\ \label{eq:bayes_map}
\tilde{\theta} =& \operatorname*{argmax}_{\theta} log (P (\mathscr{D} | \theta)) + log (P(\theta)) \nonumber \\
\nonumber \\ 
score_{MAP} (\theta : \mathscr{D}) =& \ log (P (\mathscr{D} | \theta)) + log (P(\theta)) \nonumber\\
\tilde{\theta} =& \operatorname*{argmax}_{\theta} score_{MAP}(\theta : \mathscr{D})
\end{align}

Where the last equation in (12) follows immediately from the properties of
the logarithm function. And the second equation in (12) from the fact that
the normalizing constant does not depend on the parameter of
interest.

Given the above it is possible to understand that the conclusions
from the previous chapter about the EM algorithm apply. The first
term of \(score_{MAP}\) is exactly the likelihood term of the previous
section. The only difference will be in the prior distribution term.

We will show next that it is possible to adjust the M-step of the EM
algorithm in order to have a properly working EM algorithm
maximizing the score map of \ref{eq:bayes_map}. This will be the main
exercise of the next section.

\end{article}

\subsection{Bayesian Parameter Learning - EM Generalization}
\label{sec:org53088bd}

Maximum a posteriori Bayesian Parameter Learning is a
straightforward generalization of the discussion of \ref{math_em}.

In fact noting that the score of the MAP estimator is defined as

\begin{equation} 
score_{MAP} (\theta : \mathscr{D}) =& \ log (P (\mathscr{D} | \theta)) + log (P(\theta)) 
\end{equation}

it is possible to see that the previous results apply.

In order to see that define the following adjusted energy
functional:

\begin{equation} \label{eq:adj_energy_functional}
\tilde{F}[\theta, Q] = E_Q[log (\tilde{P}(X))] + H_Q (X) + log (P(\theta)) 
\end{equation}

Such that:

\begin{align} \label{eq:adj_likelihood_energy_functional_relation}
l (\theta: \mathscr{D}) + log (P(\theta)) =& \ \tilde{F}_D[\theta, Q] + D (Q (\mathscr{H}) || P (\mathscr{H}| \theta, \mathscr{D})) 
\end{align}

It follows immediately that choosing \(Q\) as \(P (H|D, \theta)\) and
maximizing the adjusted energy functional we are in fact maximizing
the score-map such that the results of the previous section
apply. 

The only question remaining is on how to optimize the adjusted
energy functional via coordinate ascent optimization.

Here it is straightforward to see that the adjusted metric does not
affect E-step (we still choose Q in the very same way) but the
M-step needs to be reformulated taking the effect of the prior into
account.

In order to see this consider our discussion in the previous
chapter. The way you choose the Q distribution is unaffected and we
will need to perform the same step in order to get the
\(\operatorname*{argmax}_{Q} \tilde{F}_D[\theta, Q]\).

However, what is affected is the optimization along the other
coordinate. That is the computation of
\(\operatorname*{argmax}_{\theta} \tilde{F}_D[\theta, Q]\) keeping Q
fixed. In this case the terms depending on \(\theta\) is not limited to
the expected likelihood \(E_Q[l (\theta: \mathscr{D}, \mathscr{H})]\)
as was the case before but it is rather important to also consider
the prior distribution \(P(\theta)\).

\subsection{Bayesian Parameter Learning - An example}
\label{sec:orge58cdb1}

An example for the extension of the EM algorithm to compute the
maximum a posteriori parameter in the case of missing evidence is
treated in this section.

The theory proceeds with the most classic network structure. The
one of table conditional probability distributions where the
realizations are distributed according to a multinomial
distribution given the \(\theta\)\textsubscript{X\textsubscript{i} | Pa\textsubscript{X\textsubscript{i}}} local parameters and
where possible realizations are binary, \(Val(X_i) = \{0,1 \}\).

Specifying a Dirichlet distribution as the prior of such parameters
we can compute the maximum a posteriori estimator.

As from the reasoning of the previous chapter we know that the EM
algorithm properties of convergence and correctness apply and that
the algorithm will iteratively converge to a local maximum.

While as mentioned the E-step will be unaffected by the
introduction of the prior, we need to adapt the M-step to account
for the influence of the latter.

Consider in this sense the unnormalized probability for the
Dirichlet-Multinomial posterior distribution:

\begin{align} \label{eq:dirichlet-multinomial-score}
P(\theta | X) = \frac{\Gamma(\sum_i x_i + 1)}{\prod_i \Gamma(x_i + 1)} \prod_i^K \theta_{x_i | Pa_i}^{x_i}  * \frac{1}{B(\alpha)} \prod_{i=1}^K \theta_{x_i | Pa_i}^{\alpha_i - 1}
\end{align}

And consider the adjusted energy functional
\ref{eq:adj_energy_functional} from which we can derive the new
likelihood expression in the case of missing evidence by defining a
new random variable \(Y\) expressing complete data observations
\((H, D)\):

\begin{align} \label{eq:dirichlet-multinomial-likelihood}
\tilde{F}[\theta, Q] =& \ E_Q[P_\theta(Y)] + H_Q (Y)
\end{align}

Such that taking the argument maximizing the likelihood of the
adjusted energy functional \(\operatorname*{argmax}_{\theta}
   \tilde{F}[\theta, Q]\) we are left with the following with y[m] as
synthetically created complete observation <h[m], d[m]>:

\begin{align} \label{eq:first-order-condition}
\tilde{\theta} =& \operatorname*{argmax}_{\theta} \sum_m E_Q[log(\frac{\Gamma(\sum_i y[m]_i + 1)}{\prod_i \Gamma(y[m]_i + 1)} \prod_i^K \theta_{y_i | Pa{y_i}}^{y[m]_i} * \frac{1}{B(\alpha)} \prod_{i=1}^K \theta_{y_i | Pa{y_i}}^{\alpha_i - 1})] + H_Q (y[m]) \\
\nonumber\\   
\tilde{\theta} =& \operatorname*{argmax}_{\theta} \sum_m E_Q[log(\prod_i^K \theta_{y_i | Pa{y_i}}^{y[m]_i} * \theta_{y_i | Pa{y_i}}^{\alpha_i - 1})]\\
\nonumber\\   
\tilde{\theta} =& \operatorname*{argmax}_{\theta} \sum_m E_Q[log(\prod_i^K \theta_{y_i | Pa{y_i}}^{y[m]_i + \alpha_i - 1})] 
\end{align}

It follows given that by the linearity of the expectation and that
\(y[m]_i = \{0,1\}\), we can re-express the above as:

\begin{align} \label{eq:solution1}
\tilde{\theta} =& \operatorname*{argmax}_{\theta} \sum_i^K (\sum_m^M E_Q[M[y_i, Pa_{y_i}]] + \alpha_i - 1) * log(\theta_{y_i | Pa{y_i}})] 
\end{align}

where it holds

\begin{align} \label{eq:expected_sufficient}
\bar{M}[y_i, Pa_{y_i}]  =& \sum_m^M E_Q[M[y_i, Pa_{y_i}]]\\
\bar{M}[y_i, Pa_{y_i}]  =& \sum_m^M \sum_{h[m] \in Val(\mathscr{H}[m])} Q(h[m]) \mathbbm{1}_{\{Y[m]_i = y[m]_i\}}\\
\bar{M}[y_i, Pa_{y_i}]  =& \sum_m^M P(y_i | d[m], \theta)
\end{align}

So that ultimately:

\begin{align} \label{eq:solution2}
\tilde{\theta} =& \operatorname*{argmax}_{\theta} \sum_i^K (\bar{M}[y_i, Pa_{y_i}] + \alpha_i - 1) * log(\theta_{y_i | Pa{y_i}})] 
\end{align}

Given the additional restriction that \(\sum_i \theta_{y_i |
   Pa{y_i}} = 1\), we can obtain the necessary condition for finding
the optimum by using the Lagrange method

\begin{align} \label{eq:first-order1}
\frac{\partial}{\partial \theta_{y_i | Pa{y_i}}} \sum_i^K (\bar{M}[y_i, Pa_{y_i}] + \alpha_i - 1) * log(\tilde{\theta}_{y_i | Pa{y_i}})] - \lambda (\sum_i \tilde{\theta}_{y_i | Pa{y_i}} - 1) \mathrel{\stackon[5pt]{$=$}{$\scriptstyle!$}} 0
\end{align}
\begin{align} \label{eq:first-order2}
\lambda = \frac{\bar{M}[y_i, Pa_{y_i}] + \alpha_i - 1}{\tilde{\theta}_{y_i | Pa{y_i}}}
\end{align}

And inserting this in the first order condition and solving for
\(\tilde{\theta}_{y_i | Pa{y_i}}\)

\begin{align} \label{eq:solution}
\tilde{\theta}_{y_i | Pa{y_i}} =& \frac{\bar{M}[y_i, Pa_{y_i}] + \alpha_i - 1}{\sum_j \bar{M}[y_j, Pa_{y_j}] + \alpha_j - 1}
\end{align}

This will be the way you update the parameters in the M-step.

It is straightforward to see from the above that it is possible to
perform the same exercise in similar settings and as long as the
prior distribution \(P(\theta)\) is well behaved such that we can
compute the first derivative of it and get to an analytic solution
for the M-step the correctness and convergence properties of EM
apply to the score of the maximum a posteriori point estimate such
that we will choose a local maximum point estimator.


\newpage

\bibliography{../literature/references}
\bibliographystyle{unsrt}


\subsection{TODOs}
\label{sec:orgfe0e29d}

\subsubsection{{\bfseries\sffamily TODO} check if particle formulation in energy functional ok as such}
\label{sec:orge914909}

\subsubsection{{\bfseries\sffamily TODO} make more explicit the citation to koller and friedman in the chapter about the mathematics of the EM algo}
\label{sec:org6de11a2}
\end{document}
